\documentclass[a4paper,10pt]{report}
%\usepackage[sc]{mathpazo}
%\usepackage[light,firsttwo,outline,bottomafter]{draftcopy}
%\usepackage{srcltx}
\usepackage{anysize} % Soporte para el comando \marginsize
\marginsize{1.2cm}{1.2cm}{1cm}{1cm}
\usepackage{amsfonts}
\usepackage{amssymb}
\usepackage[latin1]{inputenc}
\usepackage[english,spanish]{babel}
\usepackage{amsmath}
\usepackage{multicol} 
\columnsep=7mm
\usepackage{latexsym}
\usepackage{mathrsfs}
\usepackage{indentfirst}
\usepackage{graphicx}
\usepackage{enumitem}
%\usepackage{hyperref}

\usepackage{times}



%\usepackage{xcolor}
%\usepackage{pgfplots}
%\usepackage{tikz}

% Define bar chart colors
%
%\definecolor{bblue}{HTML}{4F81BD}
%\definecolor{rred}{HTML}{C0504D}
%\definecolor{ggreen}{HTML}{9BBB59}
%\definecolor{ppurple}{HTML}{9F4C7C}





\setlength{\paperwidth}{216mm} \setlength{\paperheight}{230mm}
\setlength{\textwidth}{39pc} \setlength{\textheight}{57.5pc}
\setlength{\topmargin}{-1.5cm} \setlength{\oddsidemargin}{-0.5cm}
\setlength{\evensidemargin}{0.9cm}
%\setlength{\footskip}{-0.3cm}

\newcommand{\ds}{\displaystyle}
\newcommand{\normal}{\triangleleft \,}
\newcommand{\tx}{\textrm}
% \linespread{1.2} \sloppy


\newcommand{\Z}{\mathbb{Z}}
\newcommand{\N}{\mathbb{N}}
\newcommand{\R}{\mathbb{R}}
\newcommand{\PR}{\mathbb{P}}
\newcommand{\e}{\rightarrow}
\newcommand{\bi}{\Leftrightarrow}
\newcommand{\com}{\mathbb{N} \bi}
\newcommand{\fu}{f:\N \e \R}
\newcommand{\ba}{\backslash}
\newcommand{\Q}{\mathbb{Q}}


\newcommand{\calP}{\mathcal{P}}
\newcommand{\calF}{\mathcal{F}}
\newcommand{\calL}{\mathcal{L}}


\newcommand{\ovl}{\overline}
\newcommand{\ora}{\overrightarrow}
\newcommand{\ola}{\overleftarrow}
\newcommand{\olra}{\overleftrightarrow}
\newcommand{\ula}{\underleftarrow}
\newcommand{\ura}{\underrightarrow}


\newcommand{\inner}[2]{\langle{#1},{#2}\rangle}

%%%%%%%%%%%%%%%%%%%%%%%%%%%%%%%%

\newcommand{\cabecera}[1]{\begin{figure}[h]
 \begin{minipage}[c]{0.05\columnwidth}
\centering\includegraphics[width=2cm]{escudo.pdf}% tb escudouni.bmp
\end{minipage}
\hfill{}
\begin{minipage}[c]{0.86\columnwidth}
\centering\flushleft {Universidad Nacional de Ingenier\'ia\\
Facultad de Ciencias\\
Escuela Profesional de Matem\'atica \hfill #1}
\end{minipage}
\end{figure}\vspace{-0.5cm}
}


\pagestyle{empty}

\begin{document}
%\cabecera{Ciclo 2015-I}
\begin{center}
{\Large {\bf Principio de Inclusi\'on-Exclusi\'on}}

\end{center}
%{\normalsize  Conceptos de Probabilidad, Probabilidad condicional, Teorema de la Probabilidad Total, Teorema de Bayes}
\setlength{\unitlength}{1in}

\begin{picture}(6,.1) 
\put(0,0) {\line(1,0){6.25}}         
\end{picture}

 

\renewcommand{\arraystretch}{2}

\vskip.15in
%\noindent\textbf{Instructor:} Instructor name,  Office building and number, Phone: 555-5555
%\vskip.25in
%\noindent {\Large \textbf{Soluci\'on a los Ejercicios }}
%\vskip.25in


(\textbf{Principio de inclusi\'on-exclusi\'on}):
El principio de inclusi\'on-exclusi\'on, es una generalizaci\'on de la propiedad 

\[
P(A\cup B) = P(A) + P(B) -P(A \cap B)
\]

para dos eventos $A, B\subset \Omega$. Esto es equivalente a un resultado de la teoria de conjuntos

\[
|A \cup B|=|A| + |B| -|A \cap B|
\]

donde la notaci\'on $|A|$ significa el n\'umero de elementos contenidos en el conjunto $A$.



\vspace{0.3cm}


$\blacktriangleright$  Sean $A_1, A_2, \dots, A_n$ eventos donde  $n \geq 2$  se cumple que

\begin{align*}
\PR\Bigl(\bigcup _{i=1}^nA_i\Bigr) = \sum_{i}^{n}\PR (A_i) - \sum_{i < j}\PR(A_i \cap A_j)+   \sum_{i < j < k} \PR(A_i \cap A_j \cap A_k) - 
\\ \sum_{i < j < k < l}\PR(A_i \cap A_j \cap A_k \cap A_l)+   \dots + (-1)^{n + 1}\PR(A_1 \cap A_2 \cap \dots \cap A_n).
\end{align*}

\vspace{0.3cm}

donde $A_1, A_2, \dots, A_n\subset \Omega$. Esto es equivalente a un resultado de la teoria de conjuntos

\vspace{0.2cm}


\begin{align*}
\vert A_1 \cup A_2 \cup \cdots \cup A_n \vert = \sum_{i = 1}^{n}\vert A_i \vert - \sum_{i < j}^{}\vert A_i \cap A_j\vert + \sum_{i < j < k}^{}\vert A_i \cap A_j \cap A_k \vert \\
\sum_{i < j < k < l}^{}\vert A_i \cap A_j \cap A_k \cap A_l \vert  + \dots + (-1)^{n +1}\vert A_1 \cap A_2 \cap \dots A_n \vert
\end{align*}

\vspace{0.3cm}

Para la prueba, supongamos que un punto est\'a contenido exactamente en $m$ de los conjuntos $A_1, A_2, \dots A_n$, donde $m$ es un n\'umero entre $1$ y $n$. Entonces el punto es contado  $m$ veces en $\sum_{i = 1}^{n}\vert A_i \vert$, es contado $C(m,2)$ veces en $\sum_{i < j}^{}\vert A_i \cap A_j\vert$, es contado $C(m,3)$ veces en $\sum_{i < j< k}^{}\vert A_i \cap A_j \cap A_k \vert$, etc.

La notaci\'on $C(m,k)$ significa el n\'umero de maneras de	 escoger un conjunto de $k$ objetos desde un conjunto de $m$ objetos, sin repetici\'on.

\vspace{0.3cm}


Despu\'es de alcanzar $\sum_{i_1 < i_2 < \cdots i_m}\vert A_{i_1} \cap A_{i_2} \cap \cdots \cap A_{i_m}\vert$, donde el punto es contado una vez (desde que $C(m ,m) = 1$), se encuentra que el punto no es contado  en absoluto, en cualquiera que implique  la intersecci\'on de m\'as de $m$ conjuntos.

\vspace{0.3cm}


 El resultado aqu\'i  es que un punto que est\'a contenido en exactamente $m$ de los conjuntos se contar\'a $S$ veces en  $\vert A_1 \cup A_2 \cup \cdots \cup A_n \vert $, como en la ecuaci\'on anterior, donde $S$ tiene la forma de 

\[
S \equiv C(m,1)-C(m,2)+C(m, 3)-C(m,4) + \dots + (-1)^{m+1}C(m,m). 
\]

\vspace{0.3cm}

Para calcular $S$, recordamos el teorema binomial

\vspace{0.3cm}

\[
(x + y)^m = \sum_{k = 0}^{m} C(m, k)x^k y^{m -k}, \ \ \text{donde}\ \  C(m, k ) = \binom{m}{k} = \frac{m!}{k!( m - k)!}.
\]

\vspace{0.3cm}

Poniendo $x = 1$ y $y= -1$ en la ecuaci\'on anterior, tenemos $\sum_{k = 0}^{m}(-1)^kC(m, k) = 0$

\vspace{0.3cm}

Usando $C(m, 0) = 1$, se sigue que

\[
1- C(m,1) + C(m, 2) - C(m, 3) + \dots + (-1)^mC(m, m) = 0
\]

\vspace{0.3cm}

Lo cual implica que $S = 1$. Esto implica que cada punto contenido en la uni\'on de $A_1, A_2,\dots A_n$ es \mbox{contado exactamente} una vez. As\'i queda demostrado el resultado.
\end{document}